\documentclass[12pt, a4paper, twoside, titlepage]{article}
\usepackage{amsmath,mathtools}
\usepackage{bbold}
\usepackage{mathabx}
\newcommand{\ket}[1]{\ensuremath{\left|#1\right\rangle}}
\newcommand{\bra}[1]{\ensuremath{\left\langle#1\right|}}

\title{Hamiltonian of QLM as a spin system }
\author{Paolo Stornati \\
	DESY  Zeuthen \\
	}


\date{\today} 
% \date{\today} date coulde be today 
% \date{25.12.00} or be a certain date
% \date{ } or there is no date 
\begin{document}
% Hint: \title{what ever}, \author{who care} and \date{when ever} could stand 
% before or after the \begin{document} command 
% BUT the \maketitle command MUST come AFTER the \begin{document} command! 
%\maketitle

%\tableofcontents % create a table of contens 

%\newpage


\section{Hamiltonian of QLM as a spin system}

We define the Square Ice Hamiltonian

\begin{equation}
  \label{eq:Ham1}
    H = \sum_\square (-f_\square + \lambda f^2_\square)\,,\label{eq:hamiltonian}
\end{equation}
where we some over all plaquettes. The plaquette operator is defined as:
\begin{equation}\label{eq:plaquette}
       f_\square = \sigma^+_{\mu_1}\sigma^+_{\mu_2}\sigma^-_{\mu_3}\sigma^-_{\mu_4}\, +  \sigma^-_{\mu_1}\sigma^-_{\mu_2}\sigma^+_{\mu_3}\sigma^+_{\mu_4}
\end{equation}

\begin{align}
	f^2_\square= \sigma^+_{\mu_1} \sigma^-_{\mu_1} \, \sigma^+_{\mu_2} \sigma^-_{\mu_2} \, \sigma^-_{\mu_3} \sigma^+_{\mu_3} \, \sigma^-_{\mu_4} \sigma^+_{\mu_4}  \, + hc
\end{align}

If we define $p+$ and $p+$ as:
\begin{align}
	p+= \frac{ \mathbb{1} + \sigma^z }{2} \; ;	\; p-= \frac{ \mathbb{1} - \sigma^z }{2}
\end{align}

I have:

\begin{equation}\label{eq:plaquette}
	f^2_\square = p^+_{\mu_1}p^+_{\mu_2} p^-_{\mu_3} p^-_{\mu_4}\, +  p^-_{\mu_1} p^-_{\mu_2} p^+_{\mu_3} p^+_{\mu_4}
\end{equation}

\subsection{Todos}
\begin{itemize}
  \item Hamiltonian in external magnetic field, $\phi_\square\in \mathbb{R}$. 
    Therefore we define the generalized plaquette operator
    \begin{align}
      f(\phi_\square) \coloneqq u_\square e^{i\phi_\square}+ u^\dagger_\square e^{-i\phi_\square}
    \end{align}
    and plug it in \eqref{eq:Ham1}
    \begin{align}
      H =J \sum_{\square} ( - u_\square e^{\i \phi_\square} + u^\dagger_\square e^{-\i \phi_\square}) + \lambda \sum_\square (u_\square e^{i\phi_square}
    \end{align}
\end{itemize}


%\end{thebibliography}

\end{document}




















