\documentclass[11pt]{article}
\usepackage[utf8]{inputenc}	% Para caracteres en español
\usepackage{amsmath,amsthm,amsfonts,amssymb,amscd}
\usepackage{multirow,booktabs}
\usepackage[table]{xcolor}
\usepackage{fullpage}
\usepackage{lastpage}
\usepackage{enumitem}
\usepackage{fancyhdr}
\usepackage{mathrsfs}
\usepackage{wrapfig}
\usepackage{setspace}
\usepackage{calc}
\usepackage{multicol}
\usepackage{cancel}
\usepackage[retainorgcmds]{IEEEtrantools}
\usepackage[margin=3cm]{geometry}
\usepackage{amsmath}
\newlength{\tabcont}
\setlength{\parindent}{0.0in}
\setlength{\parskip}{0.05in}
\usepackage{empheq}
\usepackage{framed}
\usepackage[most]{tcolorbox}
\usepackage{xcolor}
\colorlet{shadecolor}{orange!15}
\parindent 0in
\parskip 12pt
\geometry{margin=1in, headsep=0.25in}
\theoremstyle{definition}
\newtheorem{defn}{Definition}
\newtheorem{reg}{Rule}
\newtheorem{exer}{Exercise}
\newtheorem{note}{Note}
\usepackage{amsmath}
\usepackage{physics}

\usepackage{bibentry}

\usepackage[colorlinks]{hyperref} % For hyperlinks in the PDF
\hypersetup{citecolor=blue, 
                    filecolor=blue,
                    linkcolor=black, 
                    urlcolor=blue}


\usepackage{amssymb}
\usepackage[utf8]{inputenc}
\usepackage{slashed}
\usepackage{dsfont}
\title{Entanglement entropy}
\date{}

\begin{document}
\author{Paolo Stornati}
\maketitle

\section{Von-Neumann entanglement entropy}

Let us suppose we have the ground state of a gapped model on a lattice $V$ described by the density matrix $\rho$. The von  Neumann entropy is defined as:
\begin{equation} 
S(\rho)=-Tr(\rho\log(\rho))
\end{equation}
The entropy of the ground-state is zero since it is a product state. We can not state this for subset of the system. 
We define a subset $A$ of $V$ suche that $B:=V / A $. We define the reduced density matrix of the sub-system as:
\begin{equation} \label{eq:Apartition}
\rho_A=Tr_B(\rho)
\end{equation}
We define the von-Neumann entropy of the subsystem $A$ as:
\begin{equation}
S(\rho_A)=-Tr(\rho_A\log_2(\rho_A))
\end{equation}
Now, the entropy $S(\rho_A)$ is null if the subsystems $A$ and $B$ are product states. This is not true if there is quantum correlations between $A$ and $B$. Quantum correlations can lead to non-vanishing values of $S(\rho_A)$. In fact, the von-Neumann entropy is a index of the entanglement for pure states. This is also why the von-Neumann entropy is also called entanglement entropy. 
\section{Matrix product states}

We define the wavefunction of a quantum state as \cite{Schollw_ck_2011}:

\begin{equation}  
\ket \psi = \sum{j_1,...,j_n} c_{j_1,...,j_n}  \ket{j_1} \otimes ... \otimes \ket{j_n}
\end{equation}

The matrix product state (MPS) Ansatz defines $c_{j_1,...,j_n}$ as:

\begin{equation}  
c_{j_1,...,j_n}= \sum_{\alpha , \beta, \, ... \, , \gamma } A^{1}_{\alpha , \beta , j_1} A^{2}_{ \beta , \delta ,  j_2} \, ... \, A^{N}_{  \gamma , \alpha ,j_n}
\end{equation}

We can now make a comparison between the number of parameters there are in the exact description and the MPS Ansatz. If we have a system of $n$ spin physical dimension $d$ we have $\mathcal{O}(d^n)$ parameters in the exact description. If we decide to truncate the dimension of the greek(i.e. $\alpha \, \beta \, ...$) to be at most $D$ ($D$ is usually called $bond \, dimension$ in the literature ) indexes in the definition of the MPS Ansatz we have $\mathcal{O}(n d D^2 )$ parameters. \\
\subsection{Gauge degree of freedom}

A MPS is defined univocally but the tensors $A^{(i)}$. On the other the set of tensors $A^{(i)}$ that define a generic $\Psi$ is not unique. $\Psi$ is defined as the contraction of the tensors $A^{(i)}$. 
Given a element of $Gl_D(\mathbb{C})$ $M$, a MPS is invariant under the insertions:
\begin{equation}
A^{(i)}_{\alpha , \, \beta }A^{(i+1)}_{\beta, \, \gamma } = A^{(i)}_{\alpha , \, \beta }M_{{\beta, \, \delta } } M^{-1}_{{\delta , \, \xi } }A^{(i+1)}_{\xi, \, \gamma } 
\end{equation}
At this point it is very important to remind the reader a fundamental too in linear algebra: The singular value decomposition (SVD). Given.a generic rectangular matrix $N$ is is always possible to find matrices $U, \, S, \, V$ such that:
\begin{equation}
N= U\,S\,V^\dag
\end{equation}
$U$ is a matrix containing the left singular vectors of $N$. Since $U$ has orthonormal columns it is also unitary $UU^\dagger=U^\dagger U=1$. $S$ is a diagonal matrix with non-negative entries. Those numbers $s_a$ are called the singular values of $N$. The number of non-zero singular values is the rank of $N$. $V^\dagger$ is a matrix that contains the right singular vectors. In the same way as  $U$, $V^\dagger$ has orthonormal columns it is also unitary $VV^\dagger=V^\dagger V=1$. The sigular values contained in $S$ has lot of interesting properties. Let us suppose we have a state $\ket \psi$. For any partition $A$ and $B$ of the Hilbert space in which $\ket \psi$ is defined it is always possible to write:
\begin{equation}\label{eq:bipartitionsvd}
\ket \psi = \sum_{\alpha, \, \beta} c_{\alpha, \, \beta} \ket{\alpha}_A \ket{\beta}_B
\end{equation}
If perform a SVD of the matrix $c$ in \autoref{eq:bipartitionsvd}, we can write:
\begin{equation}
\ket \psi = \sum_{\alpha, \, \beta} \sum_{s_a} U_{\alpha, \, s_a} S_{s_a, \, s_a} V^\dagger_{s_a, \, \beta} \ket{\alpha}_A \ket{\beta}_B
\end{equation}
We can absorb $U$ and $V$ in $A$ and $B$ due to their orthonormality in those spaces write:
\begin{equation}
\ket \psi =  \sum_{s_a} s_a  \ket{\alpha}_A \ket{\beta}_B
\end{equation}
In this decoposition it is trivial to derive the reduced density matrix for the sub-system $A$ in \autoref{eq:Apartition}:
\begin{equation}\label{eq:canonical_rho_a}
\rho_A =  \sum_{s_a} s^2_a  \left( \ket{\alpha} \bra{\alpha}\right)_A
\end{equation}
The von-Neumann entanglement entropy can be computed directly from here:

\begin{equation}\label{eq:derived_entropy}
S(\rho_A)=-Tr(\rho_A\log_2(\rho_A))= -\sum_a s^2_a\log_2 s^2_a
\end{equation}

\subsection{Canonical form}

Fixing those matrices $M$ or, more generally, fixing $M$ such that the MPS satisfies certain relations is referred to as \textit{fixing a gauge}. 
Two particular gauge, called \textit{canonical forms} are particularly useful when computing expectation values of operators (i.e. the Hamiltonian $H$ of a quantum system). Those gauge consist into choosing the matrices $M$ such that the tensors $A^{(i)}$ satisfy the following relations:

\begin{equation}\label{eq:left}
\sum_{\beta=1}^D\sum_{s=1}^d \left(A^{[s](i)}_{\alpha, \, \beta}\right)^* A^{[s](i)}_{\beta, \, \gamma}= \delta_{\alpha, \, \gamma}  \, \end{equation}

\begin{equation}\label{eq:right}
\sum_{\beta=1}^D\sum_{s=1}^d A^{[s](i)}_{\alpha, \, \beta} \left(A^{[s](i)}_{\beta, \, \gamma}\right)^* = \delta_{\alpha, \, \gamma} \, 
\end{equation}
If a MPS satisfies the relation in \autoref{eq:left} it is called to be in the \textit{left canonical form}. If a MPS satisfies the relation in \autoref{eq:right} it is called to be in the \textit{right canonical form}.

There is also a very useful notation introduced by \cite{Vidal_2003} that highlights the singular values of the matrices of a MPS:

\begin{equation} \label{eq:vidal_form}
\ket \psi = \sum_{s_1, \, ... \, , \, s_N} U^{s_1} S_{1}U^{s_2} S_{2} \, ... \, U^{s_N} S_{N} \ket{s_1, \, ... \, , \, s_N}
\end{equation}



\subsection{Computation of the Von-Neumann entropy in the MPS formalism}

Let us suppose we have a spin chain of local Physical dimension $d$ and $N$ sites described by a MPS in the  \textit{left canonical form}. The partition function $\rho$ is defined as:
\begin{equation}
\rho = \sum^d_{s_1, \, ... \, , s_N }\sum^d_{s'_1, \, ... \, , s'_N }Tr\left[ M^{1,\, s_1, \, s'_1} M^{2, \, s_2,\, s'_2} \, ... \, M^{N, \, s_N,\, s'_N} \right]\ket{s_1, \, ... \, , s_N} \bra{s'_1, \, ... \, , s'_N}
\end{equation}
where:
\begin{equation}
M^{i,\, s_i, \, s'_i} = A^{[s_i](i)} \otimes \left(A^{[s'_i](i)} \right)^\dag
\end{equation}
We now partition the system in two sub-system $A$ and $B$. $A$ includes all the sites up to \textit{k} and $B$ its complementary. 
$\rho(A)$, defined as in \autoref{eq:Apartition}, can now be written as:

\begin{equation} 
\tilde{\rho}_A^{[l]}=Tr_B(\rho)= \sum^d_{s_1, \, ... \, , s_l }\sum^d_{s'_1, \, ... \, , s'_l }  A^{1, \, s_1} ... A^{l, \, s_l} \rho^{(l)}_{[A]}  \left( A^{1, \, s'_1} \right)^\dag ... \left(A^{l, \, s'_l}\right)^\dag \ket{s_1, \, ... \, , s_l} \bra{s'_1, \, ... \, , s'_l}
\end{equation}
where $\rho^l_{[A]}$ is defined as:
\begin{equation}\label{eq:reduced_definition}
\rho^{(l)}_{[A]}= \sum^d_{s_{l+1}, \, ... \, , s_N }\sum^d_{s'_{l+1}, \, ... \, , s'_N }  A^{l+1, \, s_{l+1}} ... A^{N, \, s_N}  \left( A^{l+, \, s'_{l+1}} \right)^\dag ... \left(A^{N, \, s'_N}\right)^\dag
\end{equation}
\autoref{eq:reduced_definition} denotes a recursive relation:

\begin{equation}
\rho^{(l+1)}_{[A]} = A^{l+1, \, s_{l+1}} \rho^{(l)}_{[A]} \left( A^{l+, \, s'_{l+1}} \right)^\dag
\end{equation}


We can always rewrite a MPS such in the form of \autoref{eq:vidal_form}. From this form it is trivial to derive the von-Neumann entanglement entropy in the same fashion as in \autoref{eq:derived_entropy}. We just need to identify:
\begin{align}
\ket {\alpha}_A = \sum_{s_1, \, ... \, , \, s_l} U^{s_1} S_{1}U^{s_2} S_{2} \, ... \,S_{l-1} U^{s_l}  \ket{s_1, \, ... \, , \, s_l}\\
\ket {\alpha}_B = \sum_{s_1, \, ... \, , \, s_N} U^{s_{l+1}} S_{{l+1}}U^{s_{l+2}} S_{l+2} \, ... \, U^{s_N} S_{N} \ket{s_{l+1}, \, ... \, , \, s_N}
\end{align}
At this point we can write:

\begin{equation}\label{eq:canonical_rho_a}
\rho_A =  \sum_{s_a} s^2_a  \left( \ket{\alpha} \bra{\alpha}\right)_A= \sum_a \left( S_{l} \right)_{a, \,a} \left( \ket{\alpha} \bra{\alpha}\right)_A
\end{equation}

Since we have the coefficients of the reduced density matrix $\rho_A$ we can compute the von-Neumann entropy as:


\begin{equation}\label{eq:derived_entropy}
S(\rho_A)=-Tr(\rho_A\log_2(\rho_A))= -\sum_a \left( S_{l} \right)_{a, \,a}^2 \log_2 \left( S_{l} \right)_{a, \,a}^2
\end{equation}























\addcontentsline{toc}{section}{Bibliography}
%\bibliographystyle{acm}
%\bibliographystyle{apsrev}
\bibliographystyle{unsrt}


\bibliography{publications.bib}{}




\end{document}